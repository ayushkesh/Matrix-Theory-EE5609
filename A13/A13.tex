\documentclass[journal,12pt]{IEEEtran}
\usepackage{longtable}
\usepackage{setspace}
\usepackage{gensymb}
\singlespacing
\usepackage[cmex10]{amsmath}
\newcommand\myemptypage{
	\null
	\thispagestyle{empty}
	\addtocounter{page}{-1}
	\newpage
}
\usepackage{amsthm}
\usepackage{mdframed}
\usepackage{mathrsfs}
\usepackage{txfonts}
\usepackage{stfloats}
\usepackage{bm}
\usepackage{cite}
\usepackage{cases}
\usepackage{subfig}

\usepackage{longtable}
\usepackage{multirow}

\usepackage{enumitem}
\usepackage{mathtools}
\usepackage{steinmetz}
\usepackage{tikz}
\usepackage{circuitikz}
\usepackage{verbatim}
\usepackage{tfrupee}
\usepackage[breaklinks=true]{hyperref}
\usepackage{graphicx}
\usepackage{tkz-euclide}

\usetikzlibrary{calc,math}
\usepackage{listings}
    \usepackage{color}                                            %%
    \usepackage{array}                                            %%
    \usepackage{longtable}                                        %%
    \usepackage{calc}                                             %%
    \usepackage{multirow}                                         %%
    \usepackage{hhline}                                           %%
    \usepackage{ifthen}                                           %%
    \usepackage{lscape}     
\usepackage{multicol}
\usepackage{chngcntr}

\DeclareMathOperator*{\Res}{Res}

\renewcommand\thesection{\arabic{section}}
\renewcommand\thesubsection{\thesection.\arabic{subsection}}
\renewcommand\thesubsubsection{\thesubsection.\arabic{subsubsection}}

\renewcommand\thesectiondis{\arabic{section}}
\renewcommand\thesubsectiondis{\thesectiondis.\arabic{subsection}}
\renewcommand\thesubsubsectiondis{\thesubsectiondis.\arabic{subsubsection}}


\hyphenation{op-tical net-works semi-conduc-tor}
\def\inputGnumericTable{}                                 %%

\lstset{
%language=C,
frame=single, 
breaklines=true,
columns=fullflexible
}
\begin{document}
\onecolumn

\newtheorem{theorem}{Theorem}[section]
\newtheorem{problem}{Problem}
\newtheorem{proposition}{Proposition}[section]
\newtheorem{lemma}{Lemma}[section]
\newtheorem{corollary}[theorem]{Corollary}
\newtheorem{example}{Example}[section]
\newtheorem{definition}[problem]{Definition}

\newcommand{\BEQA}{\begin{eqnarray}}
\newcommand{\EEQA}{\end{eqnarray}}
\newcommand{\define}{\stackrel{\triangle}{=}}
\bibliographystyle{IEEEtran}
\raggedbottom
\setlength{\parindent}{0pt}
\providecommand{\mbf}{\mathbf}
\providecommand{\pr}[1]{\ensuremath{\Pr\left(#1\right)}}
\providecommand{\qfunc}[1]{\ensuremath{Q\left(#1\right)}}
\providecommand{\sbrak}[1]{\ensuremath{{}\left[#1\right]}}
\providecommand{\lsbrak}[1]{\ensuremath{{}\left[#1\right.}}
\providecommand{\rsbrak}[1]{\ensuremath{{}\left.#1\right]}}
\providecommand{\brak}[1]{\ensuremath{\left(#1\right)}}
\providecommand{\lbrak}[1]{\ensuremath{\left(#1\right.}}
\providecommand{\rbrak}[1]{\ensuremath{\left.#1\right)}}
\providecommand{\cbrak}[1]{\ensuremath{\left\{#1\right\}}}
\providecommand{\lcbrak}[1]{\ensuremath{\left\{#1\right.}}
\providecommand{\rcbrak}[1]{\ensuremath{\left.#1\right\}}}
\theoremstyle{remark}
\newtheorem{rem}{Remark}
\newcommand{\sgn}{\mathop{\mathrm{sgn}}}
\providecommand{\abs}[1]{\left\vert#1\right\vert}
\providecommand{\res}[1]{\Res\displaylimits_{#1}} 
\providecommand{\norm}[1]{\left\lVert#1\right\rVert}
%\providecommand{\norm}[1]{\lVert#1\rVert}
\providecommand{\mtx}[1]{\mathbf{#1}}
\providecommand{\mean}[1]{E\left[ #1 \right]}
\providecommand{\fourier}{\overset{\mathcal{F}}{ \rightleftharpoons}}
%\providecommand{\hilbert}{\overset{\mathcal{H}}{ \rightleftharpoons}}
\providecommand{\system}{\overset{\mathcal{H}}{ \longleftrightarrow}}
	%\newcommand{\solution}[2]{\textbf{Solution:}{#1}}
\newcommand{\solution}{\noindent \textbf{Solution: }}
\newcommand{\cosec}{\,\text{cosec}\,}
\providecommand{\dec}[2]{\ensuremath{\overset{#1}{\underset{#2}{\gtrless}}}}
\newcommand{\myvec}[1]{\ensuremath{\begin{pmatrix}#1\end{pmatrix}}}
\newcommand{\mydet}[1]{\ensuremath{\begin{vmatrix}#1\end{vmatrix}}}
\numberwithin{equation}{subsection}
\makeatletter
\@addtoreset{figure}{problem}
\makeatother
\let\StandardTheFigure\thefigure
\let\vec\mathbf
\renewcommand{\thefigure}{\theproblem}
\def\putbox#1#2#3{\makebox[0in][l]{\makebox[#1][l]{}\raisebox{\baselineskip}[0in][0in]{\raisebox{#2}[0in][0in]{#3}}}}
     \def\rightbox#1{\makebox[0in][r]{#1}}
     \def\centbox#1{\makebox[0in]{#1}}
     \def\topbox#1{\raisebox{-\baselineskip}[0in][0in]{#1}}
     \def\midbox#1{\raisebox{-0.5\baselineskip}[0in][0in]{#1}}
\vspace{3cm}
\title{Assignment 13}
\author{Ayush Kumar}
\maketitle
\bigskip
\renewcommand{\thefigure}{\theenumi}
\renewcommand{\thetable}{\theenumi}
Download latex-tikz codes from 
%
\begin{lstlisting}
https://github.com/ayushkesh
\end{lstlisting}
%
\section{\textbf{Problem}}
Let \vec{R[x]} denote the vector space of all real polynomial. Let $\vec{D:}\vec{R[x]}\rightarrow \vec{R[x]}$ denote the map $\vec{D}f$= $\frac{df}{dx}, \forall f $ then,\\ 
\begin{enumerate}
\item $\vec{D}$ is one-one.
\item $\vec{D}$ is onto. 
\item There exist $E: \vec{R[x]} \rightarrow \vec{R[x]}$ so that $D(E(f))$ = $f, \forall f$.
\item There exist $E:\vec{R[x]} \rightarrow \vec{R[x]}$ so that $E(D(f))$ = $f, \forall f$.
\end{enumerate}
%
\section{\textbf{Explanation}}
See Table \ref{eq:table:1}
\onecolumn
\begin{longtable}{|l|l|}
\hline
\text{Given} & \text{
     Let $\vec{D:}\vec{R[x]}\rightarrow \vec{R[x]}$ denote the map $\vec{D}f$= $\frac{df}{dx}, \forall f$}
\hline
\textbf{Statement 1} & \text{D is one-one.}\\
\hline
& \parbox{10cm}{\begin{align}
\text{$\vec{D}$ is not one-one because }\\
f_{1} \ne f_{2}\\
\implies Df_{1} = Df_{2}.\\
\text{Take $f_{1}$=x then $f_{2}$= x+1}
\end{align}}\\
& \parbox{10cm}{\begin{center}
\textbf{False Statement }
\end{center}}\\
\hline
\textbf{Statement 2} & \text{ D is onto.}\\
\hline
& \parbox{10cm}{\begin{align}
 \text{D is not onto because }\\
%\text{Take $f_{1}$=x then $f_{2}$= x+1}
 f(x) = \begin{cases}
               1               & x  $\in  \mathbb{Q}$ \\
               0               & x  $\in \mathbb{Q'}$ 
           \end{cases}
\end{align}}\\
& \parbox{10cm}{\begin{center}
\textbf{False Statement }
\end{center}}\\
\hline
\textbf{Statement 3} & \text{There exist $E: $\vec{R[x]} \rightarrow \vec{R[x]}$ so that $D(E(f))$ = $f ,\forall f$.}\\
\hline
& \parbox{10cm}{\begin{align}
 \text{Not True because,}\\
 \text{Every integrable operator is not differentiable.}
\end{align}}\\
& \parbox{10cm}{\begin{center}
\textbf{False Statement }
\end{center}}\\
\hline
\textbf{Statement 4} & \text{There exist $E:\vec{R[x]} \rightarrow \vec{R[x]}$ so that $E(D(f))$ = $f, \forall f$. }\\
\hline
& \parbox{10cm}{\begin{align}
 \text{$\exists$ an integrable operator such that $E(D(f))= f, \forall f$}
\end{align}}\\
& \parbox{10cm}{\begin{center}
\textbf{True Statement }
\end{center}}\\
\hline
\caption{Explanation}
\label{eq:table:1}
\end{longtable}
\end{document}