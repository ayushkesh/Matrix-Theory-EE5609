\documentclass[journal,12pt]{IEEEtran}
\usepackage{longtable}
\usepackage{setspace}
\usepackage{gensymb}
\singlespacing
\usepackage[cmex10]{amsmath}
\newcommand\myemptypage{
	\null
	\thispagestyle{empty}
	\addtocounter{page}{-1}
	\newpage
}
\usepackage{amsthm}
\usepackage{mdframed}
\usepackage{mathrsfs}
\usepackage{txfonts}
\usepackage{stfloats}
\usepackage{bm}
\usepackage{cite}
\usepackage{cases}
\usepackage{subfig}

\usepackage{longtable}
\usepackage{multirow}


\usepackage{enumitem}
\usepackage{mathtools}
\usepackage{steinmetz}
\usepackage{tikz}
\usepackage{circuitikz}
\usepackage{verbatim}
\usepackage{tfrupee}
\usepackage[breaklinks=true]{hyperref}
\usepackage{graphicx}
\usepackage{tkz-euclide}

\usetikzlibrary{calc,math}
\usepackage{listings}
    \usepackage{color}                                            %%
    \usepackage{array}                                            %%
    \usepackage{longtable}                                        %%
    \usepackage{calc}                                             %%
    \usepackage{multirow}                                         %%
    \usepackage{hhline}                                           %%
    \usepackage{ifthen}                                           %%
    \usepackage{lscape}     
\usepackage{multicol}
\usepackage{chngcntr}

\DeclareMathOperator*{\Res}{Res}

\renewcommand\thesection{\arabic{section}}
\renewcommand\thesubsection{\thesection.\arabic{subsection}}
\renewcommand\thesubsubsection{\thesubsection.\arabic{subsubsection}}

\renewcommand\thesectiondis{\arabic{section}}
\renewcommand\thesubsectiondis{\thesectiondis.\arabic{subsection}}
\renewcommand\thesubsubsectiondis{\thesubsectiondis.\arabic{subsubsection}}


\hyphenation{op-tical net-works semi-conduc-tor}
\def\inputGnumericTable{}                                 %%

\lstset{
%language=C,
frame=single, 
breaklines=true,
columns=fullflexible
}
\begin{document}
\onecolumn

\newtheorem{theorem}{Theorem}[section]
\newtheorem{problem}{Problem}
\newtheorem{proposition}{Proposition}[section]
\newtheorem{lemma}{Lemma}[section]
\newtheorem{corollary}[theorem]{Corollary}
\newtheorem{example}{Example}[section]
\newtheorem{definition}[problem]{Definition}

\newcommand{\BEQA}{\begin{eqnarray}}
\newcommand{\EEQA}{\end{eqnarray}}
\newcommand{\define}{\stackrel{\triangle}{=}}
\bibliographystyle{IEEEtran}
\raggedbottom
\setlength{\parindent}{0pt}
\providecommand{\mbf}{\mathbf}
\providecommand{\pr}[1]{\ensuremath{\Pr\left(#1\right)}}
\providecommand{\qfunc}[1]{\ensuremath{Q\left(#1\right)}}
\providecommand{\sbrak}[1]{\ensuremath{{}\left[#1\right]}}
\providecommand{\lsbrak}[1]{\ensuremath{{}\left[#1\right.}}
\providecommand{\rsbrak}[1]{\ensuremath{{}\left.#1\right]}}
\providecommand{\brak}[1]{\ensuremath{\left(#1\right)}}
\providecommand{\lbrak}[1]{\ensuremath{\left(#1\right.}}
\providecommand{\rbrak}[1]{\ensuremath{\left.#1\right)}}
\providecommand{\cbrak}[1]{\ensuremath{\left\{#1\right\}}}
\providecommand{\lcbrak}[1]{\ensuremath{\left\{#1\right.}}
\providecommand{\rcbrak}[1]{\ensuremath{\left.#1\right\}}}
\theoremstyle{remark}
\renewcommand{\arraystretch}{2}
\newtheorem{rem}{Remark}
\newcommand{\sgn}{\mathop{\mathrm{sgn}}}
\providecommand{\abs}[1]{\mathbf{\left\vert#1\right\vert}}
\providecommand{\res}[1]{\Res\displaylimits_{#1}} 
\providecommand{\norm}[1]{\mathbf{\left\lVert#1\right\rVert}}
%\providecommand{\norm}[1]{\lVert#1\rVert}
\providecommand{\mtx}[1]{\mathbf{#1}}
\providecommand{\mean}[1]{\mathbf{E\left[ #1 \right]}}
\providecommand{\fourier}{\overset{\mathcal{F}}{ \rightleftharpoons}}
%\providecommand{\hilbert}{\overset{\mathcal{H}}{ \rightleftharpoons}}
\providecommand{\system}{\overset{\mathcal{H}}{ \longleftrightarrow}}
	%\newcommand{\solution}[2]{\textbf{Solution:}{#1}}
\newcommand{\solution}{\noindent \textbf{Solution: }}
\newcommand{\cosec}{\,\text{cosec}\,}
\providecommand{\dec}[2]{\ensuremath{\overset{#1}{\underset{#2}{\gtrless}}}}
\newcommand{\myvec}[1]{\ensuremath{\begin{pmatrix}#1\end{pmatrix}}}
\newcommand{\mydet}[1]{\ensuremath{\begin{vmatrix}#1\end{vmatrix}}}
\numberwithin{equation}{subsection}
\makeatletter
\@addtoreset{figure}{problem}
\makeatother
\let\StandardTheFigure\thefigure
\let\vec\mathbf
\renewcommand{\thefigure}{\theproblem}
\def\putbox#1#2#3{\makebox[0in][l]{\makebox[#1][l]{}\raisebox{\baselineskip}[0in][0in]{\raisebox{#2}[0in][0in]{#3}}}}
     \def\rightbox#1{\makebox[0in][r]{#1}}
     \def\centbox#1{\makebox[0in]{#1}}
     \def\topbox#1{\raisebox{-\baselineskip}[0in][0in]{#1}}
     \def\midbox#1{\raisebox{-0.5\baselineskip}[0in][0in]{#1}}
\vspace{2cm}
\begin{center}
\huge Assignment 13\\
\large Ayush Kumar\\
\large ES17BTECH11002\\
\end{center}
\section{problem}
Let \vec{R[x]} denote the vector space of all real polynomial. Let $\vec{D:}\vec{R[x]}\rightarrow \vec{R[x]}$ denote the map $\vec{D}f$= $\frac{df}{dx}, \forall f $ then,\\
\begin{enumerate}
\item $\vec{D}$ is one-one.
\item $\vec{D}$ is onto. 
\item There exist $E: \vec{R[x]} \rightarrow \vec{R[x]}$ so that $D(E(f))$ = $f, \forall f$.
\item There exist $E:\vec{R[x]} \rightarrow \vec{R[x]}$ so that $E(D(f))$ = $f, \forall f$.
\end{enumerate}
\section{Solution}
\begin{align}
\intertext{Let,} f = \sum_{i=0}^{n}f_{i}x^{i} &= f_{0}+f_{1}x+f_{2}{x}^2+. . . . f_{n}x^{n}.
\end{align}\\
$f$ is a real polynomial with degree n. and $\vec{V}$ having dim($\vec{V}$)= n.
\begin{align}
    \intertext{Now,}\\
    D(f) = \frac{d}{dx}(f),  \forall {f}\\
    = \frac{d}{dx}\brak{f_{0} +f_{1}x+ ..... f_{n}x^n}.\\
    = f_{1}+2f_{2}x+....nf_{n}x^{n-1}.\\
    \and D\brak{f} \in \vec{W} , dim(\vec{W}) = (n-1).
     \intertext{As $\vec{W}$ is a real space of real polynomial with degree (n-1)}
\end{align}
\renewcommand{\thetable}{1}
\begin{longtable}{|l|l|}
    \hline
    Given& $\vec{D:}$\vec{R[x]}\rightarrow \vec{R[x]}$ denote the map $\vec{D}f$= $\frac{df}{dx}, \forall f$}\\
    \hline
    \textbf{Statement 1} & \text{D is one-one.}\\
    \hline
    &If $D$ is one-one $\implies \vec{N(D(f))}$ = 0\\
    &$\implies dim(\vec{N(D(f))} =0$\\
    &According to rank-Nullity theorem :- \\
    & $dim(\vec{N(D)})$ + $dim(\vec{R(D)}$ = $dim(\vec{V}$)\\
    & $\implies 0+ dim(\vec{R(D)}$ = n\\
    & \implies $dim(\vec{R(D))}$=n\\
    & As $dim(\vec{R(D)} \leq dim(\vec{W})$\\
    & $*$ $dim \vec{R(D)} = dim(\vec{W})$, when $\vec{D}$ is one-one\\
    & $*$ $dim \vec{R(D)}$ $<$ $dim(\vec{W})$, when $\vec{D}$ is not onto\\
    & So $dim(\vec{R(D))}$= n not possible
    \hline
     & \text{False Statement}\\
    \hline
    \textbf{Statement 2} & \text{D is onto.}\\
    \hline
    & $dim(\vec{R(D)}) = dim(\vec{W})$\\
    & $dim(\vec{R(D)}) = n-1$\\
    & As range  $\vec{W}$ will be same for $f$, \\
    & So there will be no nullspace\\
    & $\implies dim(\vec{N(D)})= 0$\\
    & But $dim(\vec{R(D)})= dim(\vec{V})$.
     \hline
     & \text{False Statement}\\
    \hline
\textbf{Statement 3} & \text{There exist $E: $\vec{R[x]} \rightarrow \vec{R[x]}$ so that $D(E(f))$ = $f ,\forall f$.}
    \hline
   & Let $E(f) = \int{f}dx$ \\
   & \because $E(f)= \int{\brak{f_{0} + f_{1}x + f_{2}x^2 + .. .. f_{n}x^n}}dx$\\
   &  = $f_{0}x+\frac{f_{1}x^2}{2} + ... ... \frac{f_{n}x^{n+1}}{n+1}$ + a, where a is any constant.\\
   & Now, $D(E(f))= \frac{d}{dx}(E(f))$\\
   & = $f_{0} +m f_{1}x+... f_{n}x^n$\\
   & $\implies f$
    \hline
     & \text{True Statement}\\
    \hline
\textbf{Statement 4} & \text{There exist $E:\vec{R[x]} \rightarrow \vec{R[x]}$ so that $E(D(f))$ = $f, \forall f$. }\\
\hline
    & This is Possible only when $E= D^{-1}$\\
    & and in that case $\vec{V}$ and $\vec{W}$ must be isomorphic.\\
    & $\implies D$ should be onto But from statement  2, we get $D$ is not onto.\\
    & So there will be no such $\vec{E}$ st. $\vec{E(D(f))}=f$\\
     \hline
     & \text{True Statement}\\
    \hline
    \caption{Solution}
    \label{tab:Cons}
\end{longtable}
\end{document}
