\documentclass[journal,12pt,twocolumn]{IEEEtran}
%
\usepackage{setspace}
\usepackage{gensymb}
\usepackage{siunitx}
\usepackage{tkz-euclide} 
\usepackage{textcomp}
\usepackage{standalone}
\usetikzlibrary{calc}
\newcommand\hmmax{0}
\newcommand\bmmax{0}

%\doublespacing
\singlespacing

%\usepackage{graphicx}
%\usepackage{amssymb}
%\usepackage{relsize}
\usepackage[cmex10]{amsmath}
%\usepackage{amsthm}
%\interdisplaylinepenalty=2500
%\savesymbol{iint}
%\usepackage{txfonts}
%\restoresymbol{TXF}{iint}
%\usepackage{wasysym}
\usepackage{amsthm}
%\usepackage{iithtlc}
\usepackage{mathrsfs}
\usepackage{txfonts}
\usepackage{stfloats}
\usepackage{bm}
\usepackage{cite}
\usepackage{cases}
\usepackage{subfig}
%\usepackage{xtab}
\usepackage{longtable}
\usepackage{multirow}
%\usepackage{algorithm}
%\usepackage{algpseudocode}
\usepackage{enumitem}
\usepackage{mathtools}
\usepackage{steinmetz}
\usepackage{tikz}
\usepackage{circuitikz}
\usepackage{verbatim}
\usepackage{tfrupee}
\usepackage[breaklinks=true]{hyperref}
%\usepackage{stmaryrd}
\usepackage{tkz-euclide} % loads  TikZ and tkz-base
%\usetkzobj{all}
\usetikzlibrary{calc,math}
\usepackage{listings}
    \usepackage{color}                                            %%
    \usepackage{array}                                            %%
    \usepackage{longtable}                                        %%
    \usepackage{calc}                                             %%
    \usepackage{multirow}                                         %%
    \usepackage{hhline}                                           %%
    \usepackage{ifthen}                                           %%
  %optionally (for landscape tables embedded in another document): %%
    \usepackage{lscape}     
\usepackage{multicol}
\usepackage{chngcntr}
\usepackage{amsmath}
\usepackage{cleveref}
%\usepackage{enumerate}

%\usepackage{wasysym}
%\newcounter{MYtempeqncnt}
\DeclareMathOperator*{\Res}{Res}
%\renewcommand{\baselinestretch}{2}
\renewcommand\thesection{\arabic{section}}
\renewcommand\thesubsection{\thesection.\arabic{subsection}}
\renewcommand\thesubsubsection{\thesubsection.\arabic{subsubsection}}

\renewcommand\thesectiondis{\arabic{section}}
\renewcommand\thesubsectiondis{\thesectiondis.\arabic{subsection}}
\renewcommand\thesubsubsectiondis{\thesubsectiondis.\arabic{subsubsection}}

% correct bad hyphenation here
\hyphenation{op-tical net-works semi-conduc-tor}
\def\inputGnumericTable{}                                 %%

\lstset{
%language=C,
frame=single, 
breaklines=true,
columns=fullflexible
}
%\lstset{
%language=tex,
%frame=single, 
%breaklines=true
%}
\usepackage{graphicx}
\usepackage{pgfplots}

\begin{document}


\newtheorem{theorem}{Theorem}[section]
\newtheorem{problem}{Problem}
\newtheorem{proposition}{Proposition}[section]
\newtheorem{lemma}{Lemma}[section]
\newtheorem{corollary}[theorem]{Corollary}
\newtheorem{example}{Example}[section]
\newtheorem{definition}[problem]{Definition}
%\newtheorem{thm}{Theorem}[section] 
%\newtheorem{defn}[thm]{Definition}
%\newtheorem{algorithm}{Algorithm}[section]
%\newtheorem{cor}{Corollary}
\newcommand{\BEQA}{\begin{eqnarray}}
\newcommand{\EEQA}{\end{eqnarray}}
\newcommand{\define}{\stackrel{\triangle}{=}}
\bibliographystyle{IEEEtran}
%\bibliographystyle{ieeetr}
\providecommand{\mbf}{\mathbf}
\providecommand{\abs}[1]{\ensuremath{\left\vert#1\right\vert}}
\providecommand{\norm}[1]{\ensuremath{\left\lVert#1\right\rVert}}
\providecommand{\mean}[1]{\ensuremath{E\left[ #1 \right]}}
\providecommand{\pr}[1]{\ensuremath{\Pr\left(#1\right)}}
\providecommand{\qfunc}[1]{\ensuremath{Q\left(#1\right)}}
\providecommand{\sbrak}[1]{\ensuremath{{}\left[#1\right]}}
\providecommand{\lsbrak}[1]{\ensuremath{{}\left[#1\right.}}
\providecommand{\rsbrak}[1]{\ensuremath{{}\left.#1\right]}}
\providecommand{\brak}[1]{\ensuremath{\left(#1\right)}}
\providecommand{\lbrak}[1]{\ensuremath{\left(#1\right.}}
\providecommand{\rbrak}[1]{\ensuremath{\left.#1\right)}}
\providecommand{\cbrak}[1]{\ensuremath{\left\{#1\right\}}}
\providecommand{\lcbrak}[1]{\ensuremath{\left\{#1\right.}}
\providecommand{\rcbrak}[1]{\ensuremath{\left.#1\right\}}}
\theoremstyle{remark}
\newtheorem{rem}{Remark}
\newcommand{\sgn}{\mathop{\mathrm{sgn}}}
\providecommand{\res}[1]{\Res\displaylimits_{#1}} 
%\providecommand{\norm}[1]{\lVert#1\rVert}
\providecommand{\mtx}[1]{\mathbf{#1}}
\providecommand{\fourier}{\overset{\mathcal{F}}{ \rightleftharpoons}}
%\providecommand{\hilbert}{\overset{\mathcal{H}}{ \rightleftharpoons}}
\providecommand{\system}{\overset{\mathcal{H}}{ \longleftrightarrow}}
	%\newcommand{\solution}[2]{\textbf{Solution:}{#1}}
\newcommand{\solution}{\noindent \textbf{Solution: }}
\newcommand{\cosec}{\,\text{cosec}\,}
\providecommand{\dec}[2]{\ensuremath{\overset{#1}{\underset{#2}{\gtrless}}}}
\newcommand{\myvec}[1]{\ensuremath{\begin{pmatrix}#1\end{pmatrix}}}
\newcommand{\mydet}[1]{\ensuremath{\begin{vmatrix}#1\end{vmatrix}}}
%\numberwithin{equation}{section}
\numberwithin{equation}{subsection}
%\numberwithin{problem}{section}
%\numberwithin{definition}{section}
\makeatletter
\@addtoreset{figure}{problem}
\makeatother
\let\StandardTheFigure\thefigure
\let\vec\mathbf
%\renewcommand{\thefigure}{\theproblem.\arabic{figure}}
\renewcommand{\thefigure}{\theproblem}
%\setlist[enumerate,1]{before=\renewcommand\theequation{\theenumi.\arabic{equation}}
%\counterwithin{equation}{enumi}
%\renewcommand{\theequation}{\arabic{subsection}.\arabic{equation}}
\def\putbox#1#2#3{\makebox[0in][l]{\makebox[#1][l]{}\raisebox{\baselineskip}[0in][0in]{\raisebox{#2}[0in][0in]{#3}}}}
     \def\rightbox#1{\makebox[0in][r]{#1}}
     \def\centbox#1{\makebox[0in]{#1}}
     \def\topbox#1{\raisebox{-\baselineskip}[0in][0in]{#1}}}
\vspace{3cm}
\title{Assignment-4\\}
\author{Ayush Kumar}
\maketitle
\newpage
%\tableofcontents
\bigskip
\renewcommand{\thefigure}{\theenumi}
\renewcommand{\thetable}{\theenumi}
\begin{abstract}
This document contains solution of Problem Geolin\brak{1.12}
\end{abstract}
Download latex-tikz codes from 
%
\begin{lstlisting}
https://github.com/ayushkesh/Matrix-Theory-EE5609/tree/master/A4
\end{lstlisting}
%
\section{\textbf{Question}}
Show that $\sin30\degree=\frac{1}{2}$ and $\cos30\degree=\frac{\sqrt{3}}{2}$.
\section{SOLUTION}
Consider an equilateral $\triangle{ABC}$ as shown in figure:\ref{fig:es1}. Let the angle bisector of $\angle$A intersect side $\vec{BC}$ at a point D between B and C.
%Draw a Perpendicular $\vec{AD}$ from A on the $\vec{BC}$, such that $\vec{AD}$ bisect $\angle A$.
\renewcommand{\thefigure}{1}
\begin{figure}[!ht]
    \centering
    \resizebox{7cm}{!}{\documentclass{standalone}

\usepackage{tikz,pgf} 
\usepackage{amsmath}

\begin{document}
\resizebox{10cm}{!}{
\begin{tikzpicture}
    \draw (0,0) node[anchor=north]{$B$}-- (4,6.9282) node[anchor=west]{$A$}-- (8,0) node[anchor=north]{$C$}-- cycle;
    \draw (4,6.9282) -- (4,0) node[anchor=north]{$D$};
    \draw (0,0)  (0:0.75cm) arc (0:60:0.75cm);
    \draw (30:1.15cm) node{$60^\circ$};
    \draw (4,0.5) -- (3.5,0.5) -- (3.5,0);
    \draw (4,6)  (54:6.8cm) arc (-63:-120:0.855cm);
    \draw (56:6.25cm) node{$30^\circ$};
    \begin{scope}[shift={(4,0)}]
        \draw (135:0.75cm) node[anchor=east]{$90^\circ$};
    \end{scope}
    \begin{scope}[shift={(8,0)}]
        \draw (0,0)  (180:0.75cm) arc (180:120:0.75cm) node[anchor=east]{$60^\circ$};
    \end{scope}
\end{tikzpicture}
}
\end{document}}
    \caption{Equilateral $\triangle{ABC}$}
    \label{fig:es1}
\end{figure}
\begin{align}
    \label{eq:1}\norm{\vec{A-B}}=\norm{\vec{B-C}}=\norm{\vec{A-C}}= 2\norm{\vec{B-D}}
\end{align}
To Find AD.
\begin{multline}
 \brak{\vec{B-A}}^{T}\brak{\vec{B-A}}\\\ =\brak{\vec{B-D+D-A}}^{T}\brak{\vec{B-D+D-A}}\\
 = [\brak{\vec{B-D}}^{T}+\brak{\vec{D-A}}^{T}][\brak{\vec{B-D}}+\brak{\vec{D-A}}]\\
 =\brak{\vec{B-D}}^{T}\brak{\vec{B-D}} + \brak{\vec{B-D}}^{T}\brak{\vec{D-A}} +\\ \brak{\vec{D-A}}^{T}\brak{\vec{B-D}} + \brak{\vec{D-A}}^{T}\brak{\vec{D-A}}
\end{multline}
Since Equilateral triangle is also equiangular, that is, all three internal angles are also congruent to each other and are each 60$\degree$,which gives $\angle$BDA= 90\degree. So BD is the perpendicular to AD the inner product is zero.
\begin{align}
   \brak{\vec{B-D}}^{T}\brak{\vec{D-A}}=0\\ \brak{\vec{D-A}}^{T}\brak{\vec{B-D}} =0
\end{align}
which gives
\begin{multline}
    \brak{\vec{B-A}}^{T}\brak{\vec{B-A}} =\\\ \brak{\vec{B-D}}^{T}\brak{\vec{B-D}} + \brak{\vec{D-A}}^{T}\brak{\vec{D-A}}\\
    \implies\norm{\vec{B-A}}^2=\norm{\vec{B-D}}^2 + \norm{\vec{D-A}}^2
\end{multline} 
\begin{align}
  \norm{\vec{D-A}}^2=\norm{\vec{B-A}}^2-\norm{\vec{B-D}}^2\label{eq:5}
\end{align}
Substitutiing  Eq \eqref{eq:1}
\begin{align}
\norm{\vec{D-A}}^2=\norm{\vec{B-A}}^2-\frac{1}{4}\norm{\vec{B-A}}^2\label{eq:24}\\
 \norm{\vec{D-A}}=\frac{\sqrt{3}}{2}\norm{\vec{B-A}}\\
  \implies\norm{\vec{B-A}} = \frac{2}{\sqrt{3}}\norm{\vec{D-A}}\label{eq:6}
\end{align}
Let $\vec{A}=0$. Then substituting in \eqref{eq:1} and \eqref{eq:6}
\begin{align}
    \label{eq:7}\norm{\vec{B}}=2\norm{\vec{B-D}}\\
    \label{eq:8}\norm{\vec{B}}=\frac{2}{\sqrt{3}}\norm{\vec{D}}
\end{align}
Square on both sides in \eqref{eq:7}.
\begin{align}
    \norm{\vec{B}}^2=4\norm{\vec{B-D}}^2\\
    \frac{1}{4}\norm{\vec{B}}^2=\norm{\vec{B}}^2+\norm{\vec{D}}^2-2\vec{B}^T\vec{D}\label{eq:9}
\end{align}
Square on both sides in \eqref{eq:8}.
\begin{align}
\norm{\vec{B}}^2 = \frac{4}{3}\norm{\vec{D}}^2\label{eq:10}
\end{align}
Using \eqref{eq:9} and \eqref{eq:10}
\begin{align}
    \frac{1}{3}\norm{\vec{D}}^2=\frac{4}{3}\norm{\vec{D}}^2+\norm{\vec{D}}^2-2\vec{D}^T\vec{D}\\
    \implies0=\norm{\vec{D}}^2-2\vec{B}^T\vec{D}\\
     \implies\vec{B}^T\vec{D}=\norm{\vec{D}}^2\label{eq:15}
\end{align}
Let $\theta=\angle BAD$. and Taking the inner product of sides BA and AD.
\begin{align}
    \brak{\vec{B}-\vec{A}}^T\brak{\vec{A}-\vec{D}} =
    \norm{\vec{B}-\vec{A}}\norm{\vec{A}-\vec{D}}\cos{\theta}
\end{align}
\begin{align}
    \cos{\theta}=\frac{\brak{\vec{B}-\vec{A}}^T\brak{\vec{A}-\vec{D}}}{\norm{\vec{B}-\vec{A}}\norm{\vec{A}-\vec{D}}}\label{eq:13}
\end{align}
Substitute $\vec{A}=0$ in \eqref{eq:13}
\begin{align}
    \label{eq:12}\implies\cos{\theta}=\frac{\vec{B}^T\vec{D}}{\norm{\vec{B}}\norm{\vec{D}}}
\end{align}
Using \eqref{eq:8} 
\begin{align}
     \label{eq:12}\implies\cos{\theta}=\frac{\vec{B}^T\vec{D}}{\frac{2}{\sqrt{3}}\norm{\vec{D}}\norm{\vec{D}}}
\end{align}
\begin{align}
\label{eq:12}\implies\cos{\theta}=\frac{\vec{B}^T\vec{D}}{\frac{2}{\sqrt{3}}\norm{\vec{D}}^2}
\end{align}
Substitute \eqref{eq:15} in \eqref{eq:12}
\begin{align}
\cos\theta = \frac{{\frac{\sqrt{3}}{2}\norm{\vec{D}}^2}}{\norm{\vec{D}}^2}\\
    \implies\cos\theta=\frac{\sqrt{3}}{2}\\
    \because cos30\degree = \frac{\sqrt{3}}{2}\\
    \implies\theta = 30\degree\\
    \because \cos^2\theta +\sin^2\theta = 1\\
     \sin30\degree = \sqrt{1-\cos^230\degree}\\
    \implies\sin30\degree = \frac{1}{2}.
\end{align} 
\end{document}
